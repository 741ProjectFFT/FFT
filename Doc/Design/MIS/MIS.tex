\documentclass[12pt, titlepage]{article}

\usepackage{amsmath, mathtools}
\usepackage{esvect}
\usepackage[round]{natbib}
\usepackage{amsfonts}
\usepackage{amssymb}
\usepackage{graphicx}
\usepackage{colortbl}
\usepackage{xr}
\usepackage{hyperref}
\usepackage{longtable}
\usepackage{xfrac}
\usepackage{tabularx}
\usepackage{float}
\usepackage{siunitx}
\usepackage{booktabs}
\usepackage{multirow}
\usepackage[section]{placeins}
\usepackage{caption}
\usepackage{fullpage}

\hypersetup{
bookmarks=true,     % show bookmarks bar?
colorlinks=true,       % false: boxed links; true: colored links
linkcolor=red,          % color of internal links (change box color with linkbordercolor)
citecolor=blue,      % color of links to bibliography
filecolor=magenta,  % color of file links
urlcolor=cyan          % color of external links
}

\usepackage{array}

%% Comments

\usepackage{color}

\newif\ifcomments\commentstrue

\ifcomments
\newcommand{\authornote}[3]{\textcolor{#1}{[#3 ---#2]}}
\newcommand{\todo}[1]{\textcolor{red}{[TODO: #1]}}
\else
\newcommand{\authornote}[3]{}
\newcommand{\todo}[1]{}
\fi

\newcommand{\wss}[1]{\authornote{blue}{SS}{#1}}
\newcommand{\an}[1]{\authornote{magenta}{Author}{#1}}
\newcommand{\wss}[1]{\authornote{blue}{SS}{#1}}



\newcommand{\progname}{Program Name}

\begin{document}

\title{Module Interface Specification for FFT Library}

\author{Yuzhi Zhao}

\date{\today}

\maketitle

\pagenumbering{roman}

\section{Revision History}

\begin{tabularx}{\textwidth}{p{3cm}p{2cm}X}
\toprule {\bf Date} & {\bf Version} & {\bf Notes}\\
\midrule
Date 1 & 1.0 & Notes\\
Date 2 & 1.1 & Notes\\
\bottomrule
\end{tabularx}

~\newpage

\section{Symbols, Abbreviations and Acronyms}

See SRS Documentation at \url {https://github.com/741ProjectFFT/FFT/tree/master/Doc/SRS}


\wss{Also add any additional symbols, abbreviations or acronyms}

\newpage

\tableofcontents

\newpage

\pagenumbering{arabic}

\section{Introduction}

The following document details the Module Interface Specifications for FFT (Fourier Transform Library). This library provides radix 2 and radix 3 FFT and IFFT functions. The input values will be read from files and the output will be written into a new text file. Fast Fourier transforms are widely used for many applications in engineering, science, and mathematics. FFT's importance derives from the fact that in signal processing and image processing it has made working in frequency domain equally computationally feasible as working in temporal or spatial domain. So this library will provide good service to other programs.\\

\wss{You really should figure out how to do 80 column wide text.  This will
  become important for the code and documentation that you add to Drasil.}

Complementary documents include the System Requirement Specifications
and Module Guide.  The full documentation and implementation can be
found at \url{https://github.com/741ProjectFFT}. 

\section{Notation}

\wss{You should describe your notation.  You can use what is below as
  a starting point.}

The structure of the MIS for modules comes from \citet{HoffmanAndStrooper1995},
with the addition that template modules have been adapted from
\cite{GhezziEtAl2003}.  The mathematical notation comes from Chapter 3 of
\citet{HoffmanAndStrooper1995}.  For instance, the symbol := is used for a
multiple assignment statement and conditional rules follow the form $(c_1
\Rightarrow r_1 | c_2 \Rightarrow r_2 | ... | c_n \Rightarrow r_n )$.

The following table summarizes the primitive data types used by \progname. 

\begin{center}
\renewcommand{\arraystretch}{1.2}
\noindent 
\begin{tabular}{l l p{7.5cm}} 
\toprule 
\textbf{Data Type} & \textbf{Notation} & \textbf{Description}\\ 
\midrule
character & char & a single symbol or digit\\
integer & $\mathbb{Z}$ & a number without a fractional component in (-$\infty$, $\infty$) \\
natural number & $\mathbb{N}$ & a number without a fractional component in [1, $\infty$) \\
real & $\mathbb{R}$ & any number in (-$\infty$, $\infty$)\\
complex & $\mathbb{C}$ & any number can be expressed in the form a + bi, where a and b are real numbers\\
pointer & * & a programming language object whose value refers to another value stored elsewhere in the computer memory using its memory address.\\
boolean &$\mathbb{B}$  & a programming language object whose value refers to another value stored elsewhere in the computer memory using its memory address.\\
\bottomrule
\end{tabular} 
\end{center}

\noindent
The specification of \progname \ uses some derived data types: sequences, strings, and
tuples. Sequences are lists filled with elements of the same data type. Strings
are sequences of characters. Tuples contain a list of values, potentially of
different types. In addition, \progname \ uses functions, which
are defined by the data types of their inputs and outputs. Local functions are
described by giving their type signature followed by their specification.

\section{Module Decomposition}

The following table is taken directly from the Module Guide document for this project.

\begin{table}[h!]
\centering
\begin{tabular}{p{0.3\textwidth} p{0.6\textwidth}}
\toprule
\textbf{Level 1} & \textbf{Level 2}\\
\midrule

{Hardware-Hiding} & ~ \\
\midrule

\multirow{3}{0.3\textwidth}{Behaviour-Hiding}
& Input Module\\
& FFT Calculation Module\\
& Output  Module\\
\midrule

\multirow{1}{0.3\textwidth}{Software Decision} & Array Data Structure Module\\
\bottomrule

\end{tabular}
\caption{Module Hierarchy}
\label{TblMH}
\end{table}

~\newpage

\section{MIS of Input  Module} \label{Input}

\subsection{Module}

Load

\subsection{Uses}
None
\subsection{Syntax}

\subsubsection{Exported Access Programs}

\begin{center}
\begin{tabular}{p{3.5cm} p{4cm} p{4cm} p{2cm}}
\hline
\textbf{Name} & \textbf{In} & \textbf{Out} & \textbf{Exceptions} \\
\hline
%load\_data & string & - &  ReadFileFail\\
%verify\_input & - & - & -\\
load\_data\_complex & $char^{N}( N \in \mathbb{N}$),\\ &$\mathbb{R}**$,\\ &$\mathbb{R}**$, \\ &$\mathbb{N}*$ & - & \\
load\_data\_real &$char^{N}( N \in \mathbb{N}$),\\ &$\mathbb{R}**$, \\ &$\mathbb{N}*$ & - & \\
makeComplexArray&$\mathbb{R}*$,\\ &$\mathbb{R}*$,\\ &$\mathbb{N}$ 
&$\mathbb{C}*$&\\
filename&  $char^{N}( N \in \mathbb{N}$) & - & -\\
ppRealNumber&$\mathbb{N}**$ & $\mathbb{N}**$ & -\\
ppImageNumber&$\mathbb{N}**$&$\mathbb{N}**$ & -\\
pInputSize& $\mathbb{Z}*$&  $\mathbb{Z}*$&-\\
size&$\mathbb{N}$ & -& -\\
realArray&$\mathbb{R}*$&-& -\\
imgArray& $\mathbb{R}*$&-&-\\

\hline
\end{tabular}
\end{center}

\subsection{Semantics}

\subsubsection{State Variables}
ppRealNumber: $\mathbb{N}**$ \\
ppImageNumber: $\mathbb{N}**$\\
pInputSize: $\mathbb{Z}*$\\
realArray: $\mathbb{R}*$\\
imgArray: $\mathbb{R}*$\\
\subsubsection{Access Routine Semantics}

load\_data\_complex :
\begin{itemize}
\item transition: ppRealNumber, ppImageNumber, pInputSize
\item output: $out:$= self
\item exception: exc := (\\
 FileReadFail)
\end{itemize}
load\_data\_real :
\begin{itemize}
\item transition: ppRealNumber, pInputSize
\item output: $out:$= self
\item exception: exc := (\\
 FileReadFail)
\end{itemize}
verify\_number\_input():
\begin{itemize}
\item transition: Check whether pInputSize = $2^{\mathbb{N}}$ or =  $3^{\mathbb{N}}$
\item output: warning message
\item exception: exc := (\\
(radix = 2, numberOfData $\neq$ $2^n, n\in \mathbb{N}$ $\Rightarrow$ NumberOfDataNotMatchRadixConstraint)\\
(radix = 3, numberOfData $\neq$ $3^n, n\in \mathbb{N}$ $\Rightarrow$ NumberOfDataNotMatchRadixConstraint))
\end{itemize}



\section{MIS of FFT Calculation Module} \label{Cal}

\subsection{Module}

FFT\_calculate

\subsection{Uses}
Input Module (Section\ref{Input})
\subsection{Syntax}

\subsubsection{Exported Access Programs}

\begin{center}
\begin{tabular}{p{3cm} p{4cm} p{4cm} p{2cm}}
\hline
\textbf{Name} & \textbf{In} & \textbf{Out} & \textbf{Exceptions} \\
\hline
%load\_data & string & - &  ReadFileFail\\
%verify\_input & - & - & -\\
radix2FFTCalc&$\mathbb{C}^ n,$\\&$\mathbb{N}$\\&$\mathbb{N}$& $\mathbb{C}^ n,$& \\
radix3FFTCalc&$\mathbb{C}^ n,$\\&$\mathbb{N}$\\&$\mathbb{N}$&$\mathbb{C}^ n,$& \\
fakeImageTerm&$\mathbb{N}$ &$\mathbb{R}$& \\
arrComN&$\mathbb{C}^ n,$&$\mathbb{C}^ n,$&\\
size&$\mathbb{N}$& -&\\
mark&$\mathbb{N}$& -&\\

\hline
\end{tabular}
\end{center}

\subsection{Semantics}

\subsubsection{State Variables}
arrComN: $\mathbb{C}^ n,$

\subsubsection{Access Routine Semantics}
getEvenOddArray(array, size):
\begin{itemize}
\item transition: Splite an array to a structure called EVEN\_ODD\_ARRAY which basically splite an array to new arrays by even and odd order and store them.
\item output:  a structure with sorted number
\item exception: exc := None
\end{itemize}
getThreeArray(array, size):
\begin{itemize}
\item transition: Splite an array to a structure called THREE\_ARRAY which basically splite an array to new arrays by three order and store them.
\item output:  a structure with sorted number
\item exception: exc := None
\end{itemize}
unitBF():
\begin{itemize}
\item transition: A radix 2 unit butterfly calculation
\item output:  A BTRValue structure with two values in it
\item exception: exc := None
\end{itemize}
unitBF2():
\begin{itemize}
\item transition: A radix 3 unit butterfly calculation
\item output:  A BTRValue2  structure with three values in it
\item exception: exc := None
\end{itemize}
ReOrder():
\begin{itemize}
\item transition:  Decompose the original input sequence to two sequences which were taken from all the odd orders and even orders.
And then redecompose each of those two new sequences into another  two sequences by picking out the odd terms and the even terms.
This function cannot complete the rearranging work untile each subsequence only contains two terms.\\
\item output:  new array in new order
\item exception: exc := None
\end{itemize}
ReOrder2():
\begin{itemize}
\item transition:  Decompose the original input sequence to three sequences which were taken from every three tuples sorted by first, second and third order.
And then redecompose each of those three new sequences into another three sequences by picking out the different orded terms.
This function cannot complete the rearranging work untile each subsequence only contains three terms.\\
\item output:  new array in new order
\item exception: exc := None

\end{itemize}

\wss{The calculate module should show the relevant equations from the SRS, along
  with your pseudo code description of the behaviour.}

\section{MIS of Output  Module} \label{output}

\subsection{Module}

output\_compute

\subsection{Uses}
FFT Calculation Module (Section\ref{Cal})
\subsection{Syntax}

\subsubsection{Exported Access Programs}

\begin{center}
\begin{tabular}{p{2cm} p{4cm} p{4cm} p{2cm}}
\hline
\textbf{Name} & \textbf{In} & \textbf{Out} & \textbf{Exceptions} \\
\hline
%load\_data & string & - &  ReadFileFail\\
%verify\_input & - & - & -\\
outputArray & $\mathbb{C}^ n$ or $\mathbb{R}^ n$& $\mathbb{C}^ n$ or $\mathbb{R}^ n$& \\
generateFile& $\mathbb{C}^ n$ or $\mathbb{R}^ n$& txt file&\\
txt file & $\mathbb{C}^ n$ or $\mathbb{R}^ n$&.txt&\\


\hline
\end{tabular}
\end{center}

\subsection{Semantics}

\subsubsection{State Variables}
txt file


\subsubsection{Access Routine Semantics}
outputArray():
\begin{itemize}
\item transition: get the output complex number array and print out the output numbers
\item output:  each output numbers
\item exception: exc := (\\
None)
\end{itemize}
generateFile():

\begin{itemize}
\item transition: get the output complex number array and write each complex number in format "real part \& image part" in a txt file with a given name.
\item output:  generate a new txt file with output number in it.
\item exception: exc := (\\
GenerateFileFail)
\end{itemize}



\newpage

\bibliographystyle {plainnat}
\bibliography {../../../ReferenceMaterial/References}


\newpage



\end{document}