\documentclass[12pt, titlepage]{article}

\usepackage{amsmath, mathtools}
\usepackage{esvect}
\usepackage[round]{natbib}
\usepackage{amsfonts}
\usepackage{amssymb}
\usepackage{graphicx}
\usepackage{colortbl}
\usepackage{xr}
\usepackage{hyperref}
\usepackage{longtable}
\usepackage{xfrac}
\usepackage{tabularx}
\usepackage{float}
\usepackage{siunitx}
\usepackage{booktabs}
\usepackage{multirow}
\usepackage[section]{placeins}
\usepackage{caption}
\usepackage{fullpage}

\hypersetup{
bookmarks=true,     % show bookmarks bar?
colorlinks=true,       % false: boxed links; true: colored links
linkcolor=red,          % color of internal links (change box color with linkbordercolor)
citecolor=blue,      % color of links to bibliography
filecolor=magenta,  % color of file links
urlcolor=cyan          % color of external links
}

\usepackage{array}

%% Comments

\usepackage{color}

\newif\ifcomments\commentstrue

\ifcomments
\newcommand{\authornote}[3]{\textcolor{#1}{[#3 ---#2]}}
\newcommand{\todo}[1]{\textcolor{red}{[TODO: #1]}}
\else
\newcommand{\authornote}[3]{}
\newcommand{\todo}[1]{}
\fi

\newcommand{\wss}[1]{\authornote{blue}{SS}{#1}}
\newcommand{\an}[1]{\authornote{magenta}{Author}{#1}}
\newcommand{\wss}[1]{\authornote{blue}{SS}{#1}}



\newcommand{\progname}{Program Name}

\begin{document}

\title{Module Interface Specification for FFT Library}

\author{Yuzhi Zhao}

\date{\today}

\maketitle

\pagenumbering{roman}

\section{Revision History}

\begin{tabularx}{\textwidth}{p{3cm}p{2cm}X}
\toprule {\bf Date} & {\bf Version} & {\bf Notes}\\
\midrule
Date 1 & 1.0 & Notes\\
Date 2 & 1.1 & Notes\\
\bottomrule
\end{tabularx}

~\newpage

\section{Symbols, Abbreviations and Acronyms}

See SRS Documentation at \url {https://github.com/741ProjectFFT/FFT/tree/master/Doc/SRS}

\wss{Also add any additional symbols, abbreviations or acronyms}

\newpage

\tableofcontents

\newpage

\pagenumbering{arabic}

\section{Introduction}

The following document details the Module Interface Specifications for
\wss{Fill in your project name and description}

Complementary documents include the System Requirement Specifications
and Module Guide.  The full documentation and implementation can be
found at \url{...}.  \wss{provide the url for your repo}

\section{Notation}

\wss{You should describe your notation.  You can use what is below as
  a starting point.}

The structure of the MIS for modules comes from \citet{HoffmanAndStrooper1995},
with the addition that template modules have been adapted from
\cite{GhezziEtAl2003}.  The mathematical notation comes from Chapter 3 of
\citet{HoffmanAndStrooper1995}.  For instance, the symbol := is used for a
multiple assignment statement and conditional rules follow the form $(c_1
\Rightarrow r_1 | c_2 \Rightarrow r_2 | ... | c_n \Rightarrow r_n )$.

The following table summarizes the primitive data types used by \progname. 

\begin{center}
\renewcommand{\arraystretch}{1.2}
\noindent 
\begin{tabular}{l l p{7.5cm}} 
\toprule 
\textbf{Data Type} & \textbf{Notation} & \textbf{Description}\\ 
\midrule
character & char & a single symbol or digit\\
integer & $\mathbb{Z}$ & a number without a fractional component in (-$\infty$, $\infty$) \\
natural number & $\mathbb{N}$ & a number without a fractional component in [1, $\infty$) \\
real & $\mathbb{R}$ & any number in (-$\infty$, $\infty$)\\
\bottomrule
\end{tabular} 
\end{center}

\noindent
The specification of \progname \ uses some derived data types: sequences, strings, and
tuples. Sequences are lists filled with elements of the same data type. Strings
are sequences of characters. Tuples contain a list of values, potentially of
different types. In addition, \progname \ uses functions, which
are defined by the data types of their inputs and outputs. Local functions are
described by giving their type signature followed by their specification.

\section{Module Decomposition}

The following table is taken directly from the Module Guide document for this project.

\begin{table}[h!]
\centering
\begin{tabular}{p{0.3\textwidth} p{0.6\textwidth}}
\toprule
\textbf{Level 1} & \textbf{Level 2}\\
\midrule

{Hardware-Hiding} & ~ \\
\midrule

\multirow{3}{0.3\textwidth}{Behaviour-Hiding}
& Input Computing Module\\
& FFT Calculation Module\\
& Output Computing Module\\
\midrule

\multirow{1}{0.3\textwidth}{Software Decision} & Array Data Structure Module\\
\bottomrule

\end{tabular}
\caption{Module Hierarchy}
\label{TblMH}
\end{table}

\newpage
~\newpage

\section{MIS of Input Computing Module} \label{Input} \wss{Use labels for cross-referencing}

\subsection{Module}

input\_compute

\subsection{Uses}
None
\subsection{Syntax}

\subsubsection{Exported Access Programs}

\begin{center}
\begin{tabular}{p{2cm} p{4cm} p{4cm} p{2cm}}
\hline
\textbf{Name} & \textbf{In} & \textbf{Out} & \textbf{Exceptions} \\
\hline
%load\_data & string & - &  ReadFileFail\\
%verify\_input & - & - & -\\
inputCmpt & string, $\mathbb{Z}\in\{2, 3\}$, string& - & \\
filename& string & - & -\\
radix&$\mathbb{Z} \in \{2, 3\}$ & - & -\\
data\_type& string& -&-\\

\hline
\end{tabular}
\end{center}

\subsection{Semantics}

\subsubsection{State Variables}
None

\subsubsection{Access Routine Semantics}

load\_data():

\begin{itemize}
\item transition: None
\item output:  None
\item exception: exc := (\\
FilenameCannotFound,\\
 FileReadFail)
\end{itemize}
verify\_para():
\begin{itemize}
\item transition: None
\item output:  None
\item exception: exc := (\\
radix $\not\in \{2, 3\} \Rightarrow$ RadixIsNotValid,\\
data\_type $\neq$ Para\_data\_type $\Rightarrow $ ParaNotMatchRealDataType)
\end{itemize}
verify\_data():
\begin{itemize}
\item transition: None
\item output:  None
\item exception: exc := (\\
(radix = 2, numberOfData $\neq$ $2^n, n\in \mathbb{N}$ $\Rightarrow$ NumberOfDataNotMatchRadixConstraint)\\
(radix = 3, numberOfData $\neq$ $3^n, n\in \mathbb{N}$ $\Rightarrow$ NumberOfDataNotMatchRadixConstraint))
\end{itemize}
fill\_zero():
\begin{itemize}
\item transition: None
\item output:  None
\item exception: exc := None
\end{itemize}


\section{MIS of FFT Calculation Module} \label{Input} \wss{Use labels for cross-referencing}

\subsection{Module}

FFT\_calculate

\subsection{Uses}
None
\subsection{Syntax}

\subsubsection{Exported Access Programs}

\begin{center}
\begin{tabular}{p{2cm} p{4cm} p{4cm} p{2cm}}
\hline
\textbf{Name} & \textbf{In} & \textbf{Out} & \textbf{Exceptions} \\
\hline
%load\_data & string & - &  ReadFileFail\\
%verify\_input & - & - & -\\
r2com() &$\mathbb{C}^ n$& - & \\
r2real()&$\mathbb{R}^ n$ & - & \\
r3com()&$\mathbb{C}^ n$ & - & \\
r3real()&$\mathbb{R}^ n$& -&\\

\hline
\end{tabular}
\end{center}

\subsection{Semantics}

\subsubsection{State Variables}
None

\subsubsection{Access Routine Semantics}

rearrange\_data\_sequence():
\begin{itemize}
\item transition: None
\item output:  None
\item exception: exc := None
\end{itemize}
calculate\_$\omega _{N}$\_set():
\begin{itemize}
\item transition: None
\item output:  None
\item exception: exc := None
\end{itemize}
radix2\_multiply\_calculation():
\begin{itemize}
\item transition: None
\item output:  None
\item exception: exc := None
\end{itemize}
radix2\_plus\_calculation():
\begin{itemize}
\item transition: None
\item output:  None
\item exception: exc := None
\end{itemize}
radix3\_multiply\_calculation():
\begin{itemize}
\item transition: None
\item output:  None
\item exception: exc := None
\end{itemize}
radix3\_plus\_calculation():
\begin{itemize}
\item transition: None
\item output:  None
\item exception: exc := None
\end{itemize}

\section{MIS of Output Computing Module} \label{Input} \wss{Use labels for cross-referencing}

\subsection{Module}

output\_compute

\subsection{Uses}
None
\subsection{Syntax}

\subsubsection{Exported Access Programs}

\begin{center}
\begin{tabular}{p{2cm} p{4cm} p{4cm} p{2cm}}
\hline
\textbf{Name} & \textbf{In} & \textbf{Out} & \textbf{Exceptions} \\
\hline
%load\_data & string & - &  ReadFileFail\\
%verify\_input & - & - & -\\
outputCmpt & $\mathbb{C}^ n$ or $\mathbb{R}^ n$& - & \\

\hline
\end{tabular}
\end{center}

\subsection{Semantics}

\subsubsection{State Variables}
None

\subsubsection{Access Routine Semantics}

generate\_file():

\begin{itemize}
\item transition: None
\item output:  None
\item exception: exc := (\\
GenerateFileFail)
\end{itemize}
verify\_output\_data():
\begin{itemize}
\item transition: None
\item output:  None
\item exception: exc := (\\
outputDataNumber $\neq$ inputDataNumberAfterFill0 $\Rightarrow$ DataNumberError)
\end{itemize}


\newpage

\bibliographystyle {plainnat}
\bibliography {../../../ReferenceMaterial/References}

\newpage

\section{Appendix} \label{Appendix}

\wss{Extra information if required}

\end{document}