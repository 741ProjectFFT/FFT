\documentclass{article}

\usepackage{tabularx}
\usepackage{booktabs}

\title{CAS 741: Problem Statement\\Simple Digital Signal Processing Using FFT }

\author{Yuzhi Zhao}

\date{Sep, 14, 2017}

%% Comments

\usepackage{color}

\newif\ifcomments\commentstrue

\ifcomments
\newcommand{\authornote}[3]{\textcolor{#1}{[#3 ---#2]}}
\newcommand{\todo}[1]{\textcolor{red}{[TODO: #1]}}
\else
\newcommand{\authornote}[3]{}
\newcommand{\todo}[1]{}
\fi

\newcommand{\wss}[1]{\authornote{blue}{SS}{#1}}
\newcommand{\an}[1]{\authornote{magenta}{Author}{#1}}
\newcommand{\wss}[1]{\authornote{blue}{SS}{#1}}



\begin{document}

\maketitle

\begin{table}[hp]
\caption{Revision History} \label{TblRevisionHistory}
\begin{tabularx}{\textwidth}{llX}
\toprule
\textbf{Date} & \textbf{Developer(s)} & \textbf{Change}\\
\midrule
Date1 & Name(s) & Description of changes\\
Date2 & Name(s) & Description of changes\\
... & ... & ...\\
\bottomrule
\end{tabularx}
\end{table}

Put your problem statement here.  \\
An FFT can rapidly computes DFT transformations by factorizing the DFT  matrix into a product of sparse factors. As a result, it manages to reduce the complexity of computing the DFT from 
$\mathcal{O}(\log{}n^2)$, which arises if one simple applies the definition of DFT, to 
$\mathcal{O}(n\log{}n)$ , where $\mathcal{}n$ is the data size. Because of this algorithm decreasing the amount of calculation incredibly so that the FFT is widely used in the digital signal processing, solving difference equations, fast discrete Hartley transform and etc.\\

One of this project's aim is to using FFT algorithm complete simple digital signal DFT and IDFT which can easily transform the signal from time domain to frequency domain and vice verse. These DFT helps signal analysis since an operation is hard to perform in time-domain. For example, the convolution of two signals in time domain, which corresponds to multiplication in frequency domain is not straightforward and is more complex than multiplication. Also, this software can help you get the spectrum so that you can see the output more straightforward. Moreover, an additional power density spectrum will be provided in case some people need it to do deeper analysis. This spectrum is a product from doing FFT according to relationship between self-correlation function and power spectral density function.\\

Another aim is to using FFT algorithm solve difference equations. Different equation is an important mathematical problem as well as applied to Biology, Computer Science and Economics fields. Thus, this software provides a good solution to difference equation problems by using efficient FFT algorithm.\\

By achieving the functions above, people in various fields such as people who want to analyze signal in frequency domain, people who try to solve a difference equation, can easily get the answer in seconds.�


 











Comments to you can be added, like this:

\wss{comment}

You can also leave comments for yourself, like this:

\an{comment}

\end{document}